\subsection{Purpose}
The purpose of this document is to thoroughly describe 
Data-dRiven PrEdictive FArMing in Telengana(DREAM).
It presents functional and non functional requirements of the system and its components.
Moreover it provides use cases and scenarios for the users involved.
\\This document is meant as a contractual basis for the customer and the developer.

\subsubsection{Goals}\label{section:1.1.1}
  \begin{table}[h]
        \rowcolors{1}{gray!25}{white}  
  \begin{tabular}{|p{2cm}|p{8cm}|}
    \hline
    G1 & allow policy makers to retrieve information from farmers and to evaluate their performance\\
    \hline
    G2 & allow farmers to communicate with each other\\
    \hline
    G3 & allow farmers to insert data and advices on his production\\
    \hline
    G4 & allow farmers to send request of help to Policy Makers\\
    \hline
    G5 & allow farmers to retrieve information relevant for their activity\\
    \hline
    G6 & Policy Makers and farmers should be able to consult the map of the zone ( and the informations stored in it ) with different levels of visibility\\
    \hline
    G6A & Policy Makers and farmers should be able to see the position of the farms on the map\\
    \hline
    G6B & Policy Makers should be able to see both the information about the production and the evaluation of each farm\\
    \hline
    G6C & farmers should be able to see only the type of production of a farmer by the map\\
    \hline
  \end{tabular}
\end{table}

\subsection{Scope}
The aim of the system is to acquire and combine data and 
information of farmers in Telengana. 
The system will also provide support both to Telengana's 
policy makers and farmers thanks to 
new innovative technologies.
\\
(non so se vogliamo mettere questo : The system consists in a back-end server application and in a web application front-end)
\\
Thanks to the system policy makers are able to get a complete picture of the agriculture status in the whole state.
    In order to do this,  Dreams provides information that makes them able to give incentives to those farmers who are performing well and keep track of those who needs help.
    The farmers have access to a forum on which they are able to communicate with other farmers, a forum to spread useful suggestions and to request for help to those who are having a harder time.
    The application provides a personalized page for each farmer in which they can find, based on his location and type of production, specific advice, meteorological forecasts and the condition of the soil.
    This information are already provided by Telengana's government.
    In this page they can also find several buttons, one that allows them to specify any problem that they face,
    another one to update their production trend. There is also another button to let those who are recognized as good farmers send advice to the system, so that everyone can improve their knowledge about the local farming ...
    Data concerning weather are already provided by Telengana's government,

\subsubsection{world phenomena}
\begin{itemize}
    \item farmers decide type of production
    \item weather conditions influence production
    \item agronomists visit periodically farmers
    \item agronomists respond to help requests from farmers
    \item farmers can be identified as those who are performing well or not.
    \item farmers receive some type of advantage if they are the best one in their production activity
\end{itemize}

\subsubsection{shared phenomena}

\begin{itemize}
    \item humidity of soil is measured by sensors 
    \item amount of water used by each farmer is retrieved by water irrigation system
    \item Telengana's governments collects data concerning weather forecast
    \item farmers insert data about their production in the system
    \item farmers can insert problems they face into the system
    \item farmers can answer to requests for help from other farmers
    \item farmers can discuss with each other through the system
    \item the system identifies the farmers who are performing well
\end{itemize}

\subsection{Glossary}
\subsubsection{Definitions}
\subsubsection{Acronyms}
\subsubsection{Abbreviations}

\subsection{Document Structure}
\begin{enumerate}
    \item \textbf{Introduction}\\
            This section offers an introduction and a brief overview of the system that is presented in the document. 
            It highlights the purpose of the system and the goals that are meant to be achieved with it. 
            At the end there is also a glossary that contains a list of definitions, acronyms and abbreviations.
            
    \item \textbf{Overall Description}\\
            This section starts with a product perspective that contains a description of the system's domain through a class diagram. 
            It includes also state diagrams which are used to give more details about the behavior of some objects in the model.
            The section contains also a clear description of the features offered by the system, 
            it identifies the actors involved and it describes their characteristics.
            At the end there are domain assumptions and general constraints.
            
    
    \item \textbf{Specif Requirements}\\
            This section enters into the details on how the system interacts with the external world. It describes 
            the interfaces that are required and offered through several visual mockups. 
            Moreover the section provides functional and nonfunctional requirements. Functional
            requirements are additionally described by use cases, sequence diagrams and scenarios.
            At the end the section focuses on nonfunctional requirements and various limitations that the system might face.

    \item \textbf{Formal Analysis using Alloy}\\
            This section provides the model described through Alloy language.
            
    \item \textbf{Effor Spent}\\
            This section has a record of the hours spent to complete this document.

\end{enumerate}



\subsubsection{External Interface Requirements}

\subsubsection{User Interfaces}
qui ci vanno tutti i mockups con descrizione

\subsubsection{Hardware Interfaces}
The application does not need any specific hardware requirements. 

\subsubsection{Software Interfaces}
The web app requires a computer with a web browser installed and connected to 
The system has to rely on a DBMS API. It allows the management of all the data the system 
needs in order to provide its functionalities, described in subsection ~\ref{subsection:2.2}.

\subsubsection{Communication Interfaces}
All the communications between users and Dream website are made via HTTPS.

\newpage

\subsection{Functional Requirements}
\subsubsection{Use Case Diagram}
\begin{figure}[H]
    \begin{center}
    \includegraphics[width=1\textwidth]{images/useCaseDiag.drawio.png}
    \caption{Use Case diagram.}
    \label{fig:state9}
    \end{center}
\end{figure}
\subsubsection{Use Cases Description ad Sequence Diagram}
\begin{enumerate}
    \item \textbf{Farmer Registration} 
        \begin{longtable}{p{0.26\linewidth}p{0.75\linewidth}}
            \toprule
            \textbf{Name} & \textbf{Farmer Registration} \\
            \midrule
            \textbf{Actors} & Farmer \\
            \midrule
            \textbf{Entry conditions} & The web application has started\\
            \midrule
            \textbf{Flow of events} & 
            \begin{enumerate}
                \item The farmer wants to sign up
                \item The farmer inserts email, name, password, farm's name and farm's position 
                \item The farmer clicks submit
                \item The system checks if email is unique and if all the form is correctly fill up 
                \item The system inserts the information in the data base
            \end{enumerate} \\
            \midrule
            \textbf{Exit conditions} & The farmer is signed up\\
            \midrule
            \textbf{Exceptions} & 
            \begin{itemize}
                \item If the farmer did not insert data correctly the system will send an alert and let the user do that again
                \item If the email is already present in the database the system will send an alert saying that the email already exists
            \end{itemize} \\
            \bottomrule
            \caption{\emph{Farmer Registration} use case description}
        \end{longtable}
        
            \begin{figure}[H]
                \begin{center}
                \includegraphics[width=0.7\textwidth]{sequence/FarmerRegistration.png}
                \caption{\emph{Farmer Registration} sequence diagram}
                \label{fig:sequence1}
                \end{center}
            \end{figure}

        \item \textbf{Farmer Login}
        \begin{longtable}{p{0.26\linewidth}p{0.75\linewidth}}
            \toprule
            \textbf{Name} & \textbf{Farmer Login} \\
            \midrule
            \textbf{Actors} & Farmer \\
            \midrule
            \textbf{Entry conditions} & The web application has started\\
            \midrule
            \textbf{Flow of events} & 
            \begin{enumerate}
                \item The farmer wants to log in
                \item The farmer inserts email and password
                \item The farmer clicks submit
                \item The system checks if the credentials are correct
                \item The system notifies the farmer about the correct login
            \end{enumerate} \\
            \midrule
            \textbf{Exit conditions} & The farmer has logged in\\
            \midrule
            \textbf{Exceptions} & 
            \begin{itemize}
                \item If the system does not recognize the email it will send and alert to the farmer saying that the email inserted is wrong
                \item If the password is not correct the system will notify the farmer
            \end{itemize} \\
            \bottomrule
            \caption{\emph{Farmer Login} use case description}
        \end{longtable}
        
        \begin{figure}[H]
        \begin{center}
        \includegraphics[width=0.5\textwidth]{sequence/FarmerLogin.png}
        \caption{\emph{Farmer Login} sequence diagram}
        \label{fig:sequence2}
        \end{center}
        \end{figure}


        \item Farmer send a message on the Forum
        \item \begin{longtable}{p{0.26\linewidth}p{0.75\linewidth}}
            \toprule
            \textbf{Name} & \textbf{Farmer sends a message on the forum} \\
            \midrule
            \textbf{Actors} & Farmer \\
            \midrule
            \textbf{Entry conditions} & The farmer has logged in\\
            \midrule
            \textbf{Flow of events} & 
            \begin{enumerate}
                \item The farmer wants to send a message
                \item The farmer clicks on forum button
                \item The system send the ser to the forum page
                \item The farmer inserts the message 
                \item The farmer clicks on send message
                \item The system inserts the message into the database 
            \end{enumerate} \\
            \midrule
            \textbf{Exit conditions} & The farmer's message is published\\
            \midrule
            \textbf{Exceptions} & 
            \begin{itemize}
                \item If the message body is empty the system shows an error alert
            \end{itemize} \\
            \bottomrule
            \caption{\emph{Farmer message} use case description}
        \end{longtable}
        \begin{figure}[H]
            \begin{center}
            \includegraphics[width=0.6\textwidth]{sequence/messageOnForum.png}
            \caption{\emph{Farmer message} sequence diagram}
            \label{fig:sequence3}
        \end{center}
        \end{figure}

    \item Find a farmer on the Map\\
    \begin{longtable}{p{0.26\linewidth}p{0.75\linewidth}}
        \toprule
        \textbf{Name} & \textbf{Farmer visualizes the map} \\
        \midrule
        \textbf{Actors} & Farmer \\
        \midrule
        \textbf{Entry conditions} & The farmer has logged in\\
        \midrule
        \textbf{Flow of events} & 
        \begin{enumerate}
            \item The farmer wants to visualize the map
            \item The farmer clicks on map button
            \item The system send the farmer to the map page
            \item The farmer visualizes the map and selects a farm
            \item The system retrieves the information about the farm and shows them to the farmer
            \item The farmer visualizes the data about the selected farm
        \end{enumerate} \\
        \midrule
        \textbf{Exit conditions} & The farmer visualized the map\\
        \midrule
        \textbf{Exceptions} & \\
        \bottomrule
        \caption{\emph{Farms' map visualization} use case description}
    \end{longtable}
    \begin{figure}[H]
        \begin{center}
        \includegraphics[width=0.45\textwidth]{sequence/VisializeMap.png}
        \caption{\emph{Farms' map visualization} sequence diagram}
        \label{fig:sequence4}
        \end{center}
    \end{figure}
    
    \item Find farm’s information
    \begin{longtable}{p{0.26\linewidth}p{0.75\linewidth}}
        \toprule
        \textbf{Name} & \textbf{Farmer visualizes their own data} \\
        \midrule
        \textbf{Actors} & Farmer \\
        \midrule
        \textbf{Entry conditions} & The farmer has logged in\\
        \midrule
        \textbf{Flow of events} & 
        \begin{enumerate}
            \item 
        \end{enumerate} \\
        \midrule
        \textbf{Exit conditions} & The farmer visualizes his own data\\
        \midrule
        \textbf{Exceptions} & 
        \begin{enumerate}
            \item 
        \end{enumerate}\\
        \bottomrule
        \caption{\emph{Farm information visualization} use case description}
    \end{longtable}
    \begin{figure}[H]
        \begin{center}
        \includegraphics[width=0.7\textwidth]{sequence/FarmInformation.png}
        \caption{\emph{Farm information visualization} sequence diagram}
        \label{fig:sequence5}
        \end{center}
    \end{figure}

    \item Submit a request of help\\
    \textbf{I put only the case of a forml request to the Policy Maker (button on Farm Page) if you want we can create 2 "secion" (alt) one with this formal request and one sending a message on Forum (but is yet specified in 3)}
    \begin{figure}[H]
        \begin{center}
        \includegraphics[width=0.7\textwidth]{sequence/HelpRequest.png}
        \caption{Submit a request of help.}
        \label{fig:state6}
        \end{center}
    \end{figure}
    \item Submit an advice
    \begin{figure}[H]
        \begin{center}
        \includegraphics[width=0.7\textwidth]{sequence/AdviceSubmit.png}
        \caption{Submit an advice.}
        \label{fig:state7}
        \end{center}
    \end{figure}
    \item Visualize notifications
    \begin{figure}[H]
        \begin{center}
        \includegraphics[width=0.7\textwidth]{sequence/SeeNotifications.png}
        \caption{Visualize notifications.}
        \label{fig:state8}
        \end{center}
    \end{figure}
    \item Policy Maker login
    \begin{figure}[H]
        \begin{center}
        \includegraphics[width=0.7\textwidth]{sequence/PolicyMakerLogin.png}
        \caption{Policy Maker login.}
        \label{fig:state9}
        \end{center}
    \end{figure}
    \item Find a farm's page
    \begin{figure}[H]
        \begin{center}
        \includegraphics[width=0.7\textwidth]{sequence/PolicyMakerseachFarm.png}
        \caption{Find a farm's page.}
        \label{fig:state9}
        \end{center}
    \end{figure}
    \item Find a farmer on the Map
    \begin{figure}[H]
        \begin{center}
        \includegraphics[width=0.7\textwidth]{sequence/searchOnMap.png}
        \caption{Find a farmer on the Map.}
        \label{fig:state9}
        \end{center}
    \end{figure}
    \item Update the Map
    \begin{figure}[H]
        \begin{center}
        \includegraphics[width=0.7\textwidth]{sequence/updateMap.png}
        \caption{Update the Map.}
        \label{fig:state9}
        \end{center}
    \end{figure}
    \item Reply to a request of help
    \begin{figure}[H]
        \begin{center}
        \includegraphics[width=0.7\textwidth]{sequence/replyHelp.png}
        \caption{Reply to a request of help.}
        \label{fig:state9}
        \end{center}
    \end{figure}
\end{enumerate}
\subsubsection{Scenarios}
\subsubsection{Requirements}
\textbf{R1} the system must allow farmers to register\\
\textbf{R2} the system must allow farmers to log in\\
\textbf{R3} the system must save the farmers registration data\\
\textbf{R4} the system must guarantee that each email address is unique\\
\textbf{R5} the system must verify that the email address is valid (the type!)\\
\textbf{R6} the system must save the farmers information about their production submitted\\
\textbf{R7} the system must allow farmers to insert the type of production \\
\textbf{R8} the system must allow farmers to insert the amount of production type\\
\textbf{R9} the system must allow farmers to specify a problem they faced to the Policy Makers\\
\textbf{R10} the system must allow farmers to select the type of production on which they had troubles\\
\textbf{R11} the system must save the advice submitted by the farmers\\
\textbf{R12} the system must allow farmers to select the type of product in their suggestion\\
\textbf{R13} the system must be able to show to the farmers advices send by the Policy Makers (as a notification)\\
\textbf{R14} the system must be able to show the meteorological data of the Farm’s position\\
\textbf{R15} the system must be able to show the farm’s sensor data \\
\textbf{R16} the system must allow farmers to send messages on the forum\\
\textbf{R17} the system must register date and time of a message in the forum\\
\textbf{R18} the system must be able to show all the messages on the forum\\
\textbf{R19} the system must be able to show the map of the zone\\
\textbf{R20} the system must be able to show the farms position on the map\\
\textbf{R21} the system must be able to show on the map if a farm is performing well or not \\
\textbf{R22} the system must allow farmers to visualise notification send by Policy Makers\\
\textbf{R23} the system does not allow Policy Makers to register\\
\textbf{R24} the system must allow Policy Makers to log in\\
\textbf{R25} the system must allow Policy Makers to search a farm by name (OK?)\\
\textbf{R26} the system must allow Policy Makers to see all farms’ pages\\
\textbf{R27} the system must not allow Policy Makers to modify any farm’s page\\
\textbf{R28} the system must allow Policy Makers to update the performance of a farmer\\
\textbf{R29} the system must allow Policy Makers to send notification to the farmers\\
\textbf{R30} the system must allow Policy Makers to receive request of help by the farmers\\
\subsubsection{Traceability Matrix}

\subsection{Performance Requirements}
The system serves its user with a web application. All the computations will take place on the server side, 
thus the app is meant to be lightweighted. Moreover the load in the night is expected to be really low.
There are no problems about reliability. The insertion of new data requires a quick response in order to store 
it in the system.

\subsection{Design Constraints}
\subsubsection{Standards Compliance}
The only standard that needs to be highlighted here is the interaction with the database. It is important that the information are stored 
in a standardized form. In such manner, it is easier to memorize and retrieve data.


\subsubsection{Any Other Constraints}
Interaction between Dream and users needs to consider also regulatory policies.
As a matter of fact the application asks and retrieves data of each farmer.
More information about security and privacy will be provided in the section~\ref{subsubsection:3.4.3}


\subsection{Software System Attributes}

\subsubsection{Reliability}
The system must prevent any failure in order to guarantee continuity. 
Simultaneous accesses are expected to work, especially in the afternoon when users are expectd to insrt and retrieve more information.

\subsubsection{Availability}
It is expected that the system has the lowest downtime possible. 
The system is available with a minimum time of 96\%, 
so that it about 14 days a year of downtime are allowed.


\subsubsection{Security}
\label{subsubsection:3.4.3}
Security of the data and of the communication user-system is a primary concern. Users credential are stored in a data base, so the system crypt the password data before store it. The system recognises the right type of user during the log in phase to ensure providing the correct level visibility of the data and the permission to update them or not. In this way farmer’s privacy is guaranteed. As a matter of fact a farmer can’t see another farmer’s page.


\subsubsection{Maintanability}
The web application requires ordinary maintenance for improvements and in order to fix potential bugs. 
It is going to be scheduled at local night time, when user traffic is the lowest.
Moreover the system must be designed in a way that allows future addition of features.

\subsubsection{Portability}
The system as a web application must run on different software system as Windows, Linux an macOS.
On the server side is crucial focusing on the interaction between APIs and the data base to insert, update or read data.


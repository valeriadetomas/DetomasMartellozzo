\subsection{Purpose}
This document describes the design choices that are applied to the Dream system.
This document includes the architecture of the system and the details of what functionality will be performed in the different modules. 
Moreover there is a detailed description of how the requirements and use cases specified in the 
Requirement Analysis and Specification Document will be implemented in the system.
%---------------------%
\subsection{Scope}
The aim of the system is to acquire and combine data and information of farmers in Telengana. 
The system will provide support both to Telengana’s policy makers and to farmers 
thanks to new innovative technologies.\\
Through the system, policy makers are able 
to get a complete picture of the agriculture status in the whole state. In order 
to obtain this, Dream provides information that makes policy makers able to give 
incentives to those farmers who are performing well. Moreover it allows them to 
keep track of those farmers who need help. \\
The farmers have access to a forum where they are able to communicate with other 
farmers. The aim of the forum is to share useful suggestions and to let farmers 
who struggle with something ask for help. \\
Therefore on the one side policy makers can have an entire perspective of the farms’ 
situation in the entire state and on the other side farmers can take advantage of the 
application and discuss with their colleagues.
\par 
Hence the application is used by the policy makers as a way to monitor farmers. 
Through the system they are able to search a farm and have access to its general information. 
Once a month they also send each farmer an evaluation message where they specify if the farmer activity was
performed good or bad.\\
On the contrary farmers can ask for help or write advices. They are free to write messages and communicate with other farmers through a forum 
that is created specifically for them.
%---------------------%
\subsection{Definitions, Acronyms, Abbreviations}
\subsubsection{Definitions}
\begin{itemize}
        \item \textbf{DBMS} is a software that works as an interface between the end user and the database. It manages the data, the database engine and the database schema.
        \item \textbf{HTTPS} is a protocol where encrypted HTTP data is transferred over a secure connection. It also guarantees the privacy and integrity of data.
        \item \textbf{API} is a programming code that helps communicate two different computer programmes.
        \item \textbf{Thin Client} is a virtual desktop computing model that runs on the resources stored on a central server instead of a computer's resources
        \item \textbf{Tier} is used to describe physical distribution of components of a system
\end{itemize}
\subsubsection{Acronyms}
\begin{itemize}
        \item \textbf{Dream}: Data-Driven Predictive Farming
        \item \textbf{API}: Application Programming Interface
        \item \textbf{HTTPS}: Hypertext Transfer Protocol Secure
        \item \textbf{DBMS}: Data Base Management System
        \item \textbf{ER}: Entity Relationship Diagram
        \item \textbf{REST}: Representational State Transfer
        \item \textbf{GUI}: Graphical User Interface
        \item \textbf{RASD}: Requirement Analysis and Specification Document
\end{itemize}
\begin{itemize}
    \item \textbf{ID} Identifier
    \item \textbf{Rn} requirement number n
\end{itemize}

%---------------------%
\subsection{Revision history}
\begin{itemize}
    \item January 9, 2021: version 1.0 (First release)
\end{itemize}

%---------------------%
\subsection{Reference Documents}
\begin{itemize}
    \item Specification document: R\&DD Assignment A.Y. 2021-2022
    \item Valeria Detomas and Sofia Martellozzo. \textsl{Dream: Requirements Analysis and Specification Document.}. Software Engineering 2 Project, 2021.
\end{itemize}

%---------------------%
\subsection{Document Structure}
\begin{enumerate}
    \item \textbf{Introduction}\\
            This section offers an introduction and a brief overview of the system that is presented in the document. 
            It highlights the purpose of the document.
            At the end there is also a glossary that contains a list of definitions, acronyms and abbreviations.
            
    \item \textbf{Architectural Design}\\
            This section starts with a high level overview of how the system is divided into subsystems. It identifies each subsystem
            and assigns a role to each of them. 
            The section contains also a clear description of the how the subsystems communicate with each other in order to 
            be able to do the features required.
            
    \item \textbf{User Interface Design}\\
            This section enters into the details on how the system interacts through the user’s perspective. 
            It shows how the user will be able to interact with the system to use all the features offered.
            It describes the features offered through Ux diagram, which describe the flow of the various mockups. 
            
    \item \textbf{Requirement Traceability}\\
        In this section all the requirements previously discussed in the RASD file will be matched to the components 
        in order to specify how components are involved in addressing each requirement.
            
    \item \textbf{Implementation, Integration and Test Plan}\\
        This section starts with the description of how the implementation is going to be like. 
        Moreover there is an explanation of how the technologies are going to be used.
            
    \item \textbf{Effor Spent}\\
            This section has a record of the hours spent to complete this document.

\end{enumerate}
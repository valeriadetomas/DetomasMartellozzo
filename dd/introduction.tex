\subsection{Purpose}
This document describes the design choices that are applied to the Dream system.
This document includes the architecture of the system and the details of what functionality will be performed in the different modules. 
Moreover there is a detailed description of how the requirements and use cases specified in the 
Requirement Analysis and Specification Document will be implemented in the system.
%---------------------%
\subsection{Scope}
The aim of the system is to acquire and combine data and information of farmers in Telengana. 
The system will provide support both to Telengana’s policy makers and to farmers 
thanks to new innovative technologies.\\
Through the system, policy makers are able 
to get a complete picture of the agriculture status in the whole state. In order 
to obtain this, Dream provides information that makes policy makers able to give 
incentives to those farmers who are performing well. Moreover it allows them to 
keep track of those farmers who need help. \\
The farmers have access to a forum where they are able to communicate with other 
farmers. The aim of the forum is to share useful suggestions and to let farmers 
who struggle with something ask for help. \\
Therefore on the one side policy makers can have an entire perspective of the farms’ 
situation in the entire state and on the other side farmers can take advantage of the 
application and discuss with their colleagues.
\par 
Hence the application is used by the policy makers as a way to monitor farmers. 
Through the system they are able to search a farm and have access to its general information. 
Once a month they also send each farmer an evaluation message where they specify if the farmer activity was
performed good or bad.\\
On the contrary farmers can ask for help or write advices. They are free to write messages and communicate with other farmers through a forum 
that is created specifically for them.
%---------------------%
\subsection{Definitions, Acronyms, Abbreviations}

%---------------------%
\subsection{Revision history}

%---------------------%
\subsection{Reference Documents}
\begin{itemize}
    \item Specification document: R\&DD Assignment A.Y. 2021-2022
    \item alloytool.org : Alloy Documentation
\end{itemize}

%---------------------%
\subsection{Document Structure}
\begin{enumerate}
    \item \textbf{Introduction}\\
            This section offers an introduction and a brief overview of the system that is presented in the document. 
            It highlights the purpose of the document.
            At the end there is also a glossary that contains a list of definitions, acronyms and abbreviations.
            
    \item \textbf{Architectural Design}\\
            This section starts with a product perspective that contains a description of the system's domain through a class diagram. 
            It includes also state diagrams which are used to give more details about the behavior of some objects in the model.
            The section contains also a clear description of the features offered by the system, 
            it identifies the actors involved and it describes their characteristics.
            At the end there are domain assumptions and general constraints.
            
    
    \item \textbf{User Interface Design}\\
            This section enters into the details on how the system interacts with the external world. It describes 
            the interfaces that are required and offered through several visual mockups. 
            Moreover the section provides functional and nonfunctional requirements. Functional
            requirements are additionally described by use cases, sequence diagrams and scenarios.
            At the end the section focuses on nonfunctional requirements and various limitations that the system might face.

    \item \textbf{Requirement Traceability}\\
            This section provides the model described through Alloy language.
            
    \item \textbf{Implementation, Integration and Test Plan}\\
            
    \item \textbf{Effor Spent}\\
            This section has a record of the hours spent to complete this document.

\end{enumerate}
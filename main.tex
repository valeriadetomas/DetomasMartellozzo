\documentclass{article}
\usepackage[utf8]{inputenc}

\title{RASD_Proj}
\author{sofia.martellozzo }
\date{November 2021}

\begin{document}

\maketitle

\newpage

\tableofcontents


\newpage


\section{Introduction}

The purpose of the software is to improve the agriculture sector in India, especially in Telengana, developing new innovative technologies.
Dream retrieves different kind of data, combining them in aim to give useful and important information both to Telengana's policy makers and farmers.

    Thanks to this system policy makers are able to get a complete picture of the agriculture status in the whole state.
    In order to do this,  Dreams provides information that makes them able to give incentives to those farmers who are performing well and keep track of those who needs help.
    The farmers have access to a forum on which they are able to communicate with other farmers, a .. to spread useful suggestions and to request for help to those who are having a harder time.
    The application provides a personalized page for each farmer in which they can find, based on his location and type of production, specific advice, meteorological forecasts and the condition of the soil.
    This information are already provided by Telengana's government.
    In this page they can also find several buttons, one that allows them to specify any problem that they face,
    another one to update their production trend. There is also another button to let those who are recognized as good farmers send advice to the system, so that everyone can improve their knowledge about the local farming ...
    Data concerning weather are already provided by Telengana's government, 
    

\section{world phenomena}
- farmers decide type of production
- weather conditions influence production
- agronomists visit periodically farmers
- agronomists respond to help requests from farmers
- farmers can be identified as those who are performing well or not.
- farmers receive some type of advantage if they are the best one in their production activity

\section{shared phenomena}
- humidity of soil is measured by sensors 
- amount of water used by each farmer is retrieved by water irrigation system
- Telengana's governments collects data concerning weather forecast
- farmers insert data about their production in the system
- farmers can insert problems they face into the system
- farmers can answer to requests for help from other farmers
- farmers can discuss with each other through the system
- the system identifies the farmers who are performing well


\section{goal}
- allow policy makers to retrieve information from farmers
- allow farmers to communicate with each other
- allow farmers to insert data, questions, problems
- the impact of meteorological data on farmers activity can be used for further information
- allow farmers to retrieve information relevant for their activity (meteo, humidity..)

\section{SECTION 2}
\subsection{Class Diagram}
\subsection{State Diagram}
\subsection{Product Functions}
- insertion of data of production
- identify how farmers are performing
- farmers visualize relevant data 
- insertion of thing (request, suggestion, answer) in the forum
- interaction between farmers

\subsubsection{Scenario}
ti aggiorni automaticamente?

\end{document}


\documentclass{article}
\usepackage[english]{babel}
\usepackage[utf8]{inputenc}
\usepackage{tikz}
\usepackage{graphicx}
\usepackage{float}




\title{RASDproj}

\date{Prof. Elisabetta Di Nitto  -  Anno 2021/2022}

\author{Sofia Martellozzo - 
      Valeria Detomas 
}

\begin{document}
\maketitle
\newpage
\renewcommand\contentsname{Contents}
\tableofcontents

\newpage

\section{Introduction}

\subsection{Purpose}
The purpose of this document is to thoroughly describe 
Data-dRiven PrEdictive FArMing in Telengana(DREAM).
It presents functional and non functional requirements of the system and its components.
Moreover it provides use cases and scenarios for the users involved.
\\This document is meant as a contractual basis for the customer and the developer.

\subsubsection{Goals}
\label{section:1.1.1}
\begin{itemize}
    \item  allow policy makers to retrieve information from farmers
\item allow farmers to communicate with each other
\item allow farmers to insert data, questions, problems
\item the impact of meteorological data on farmers activity can be used for further information
\item allow farmers to retrieve information relevant for their activity (meteo, humidity..)
\end{itemize}

\subsection{Scope}
The aim of the system is to acquire and combine data and 
information of farmers in Telengana. 
The system will also provide support both to Telengana's 
policy makers and farmers thanks to 
new innovative technologies.
\\
(non so se vogliamo mettere questo : The system consists in a back-end server application and in a web application front-end)
\\
Thanks to the system policy makers are able to get a complete picture of the agriculture status in the whole state.
    In order to do this,  Dreams provides information that makes them able to give incentives to those farmers who are performing well and keep track of those who needs help.
    The farmers have access to a forum on which they are able to communicate with other farmers, a forum to spread useful suggestions and to request for help to those who are having a harder time.
    The application provides a personalized page for each farmer in which they can find, based on his location and type of production, specific advice, meteorological forecasts and the condition of the soil.
    This information are already provided by Telengana's government.
    In this page they can also find several buttons, one that allows them to specify any problem that they face,
    another one to update their production trend. There is also another button to let those who are recognized as good farmers send advice to the system, so that everyone can improve their knowledge about the local farming ...
    Data concerning weather are already provided by Telengana's government,

\subsubsection{world phenomena}
\begin{itemize}
    \item farmers decide type of production
    \item weather conditions influence production
    \item agronomists visit periodically farmers
    \item agronomists respond to help requests from farmers
    \item farmers can be identified as those who are performing well or not.
    \item farmers receive some type of advantage if they are the best one in their production activity
\end{itemize}

\subsubsection{shared phenomena}

\begin{itemize}
    \item humidity of soil is measured by sensors 
    \item amount of water used by each farmer is retrieved by water irrigation system
    \item Telengana's governments collects data concerning weather forecast
    \item farmers insert data about their production in the system
    \item farmers can insert problems they face into the system
    \item farmers can answer to requests for help from other farmers
    \item farmers can discuss with each other through the system
    \item the system identifies the farmers who are performing well
\end{itemize}

\subsection{Glossary}
\subsubsection{Definitions}
\subsubsection{Acronyms}
\subsubsection{Abbreviations}






\section{Overall Description}
\subsection{Product Perspective}
\subsubsection{Class Diagram}
\begin{figure}[H]
    \begin{center}
    \includegraphics[width=1\textwidth]{images/UMLSW2_1.png}
    \caption{UML diagram.}
    \label{fig:uml}
    \end{center}
\end{figure}
\subsubsection{State Diagram}
\begin{figure}[H]
    \begin{center}
    \includegraphics[width=1\textwidth]{images/State chart 1.png}
    \caption{Update Farmer page.}
    \label{fig:state1}
    \end{center}
\end{figure}
\begin{figure}[H]
    \begin{center}
    \includegraphics[width=1\textwidth]{images/State chart 2.png}
    \caption{Analysis of farmers.}
    \label{fig:state2}
    \end{center}
\end{figure}
\begin{figure}[H]
    \begin{center}
    \includegraphics[width=1\textwidth]{images/State chart 3.png}
    \caption{Request of help.}
    \label{fig:state3}
    \end{center}
\end{figure}
\newpage
\subsection{Product Functions}
\label{subsection:2.2}
This section provides a summary of the main features and 
functions offered by the software regarding
the goals already described in section ~\ref{section:1.1.1}

In the following description it is important to highlight that both
the policy makers and the farmers must be logged in.

\subsubsection{Farmers insert data} 
This functionality is accessible to all farmers. 
The application provides a form in which the farmer can easily insert data of his/her production.
The form is easy to fill in, in order to complete it the farmer need to indicate:
\begin{itemize}
    \item the \textbf{type of product}
    \item the \textbf{amount} producted of the selected type
    \item the \textbf{date} relative to the date of the production
\end{itemize}
If the farmer needs to add more than one type of product, 
he/she can fill in the form multiple times.
After completing the form the user is redirected to the homepage and the policy makers 
can see the updated data.
This functionality can be done more than once a day since the farmer can select the date, 
so it is possible for him to insert data of past days too.
This operation can be repeated more than once a day since the farmer can select the date, so it is possible for him/her to insert data of past days too.



\subsubsection{Farmers visualize data}
This functionality lets the farmer visualize all the data 
acquired from the system. The farmer can visualize all data on his homepage.\\
The application shows:
\begin{itemize}
    \item meteorological  short-term and long-term forecast
    \item amount of water used by the farmer
    \item humidity of soil 
    \item personalized suggestions concerning specific crops to plant or specific 
    fertilizers to use – based on their location and type of production
\end{itemize}

\textbf{Should I say why he needs to visualize this data or how he uses it}


This functionality is always up to date, and does not need 
any input from the farmer.
It is used by the farmer only to have a general view of 
his farm and on how he could improve the productivity of his farm.



\subsubsection{Identify how farmers are performing}
The main features of the policy makers is to evaluate the work of each Farmer. 
In order to do that, they periodically analyze each farm page: 
the system allows them to visualize all the data in those pages 
(but not to modify it).
The analysis takes place twice a month.
With this analysis they classify the workers in two different way:
\begin{itemize}
    \item GOOD farmer : those how have been able to produce a significant amount of product with low resources, despite bad weather in that period.
    \item BAD farmer : those who did not produce much.
\end{itemize}
Policy makers inform each farmer the result they have achieved with a notification: 
\begin{itemize}
    \item GOOD farmers receive a special 
    incentive, and also a request to submit 
    from their personal web page some advice that could be useful to the others. 
    \item To BAD farmers is asked to submit an explicit request of help specifying the problems they had.
\end{itemize}
The system has a specific web page that allows all 
the users to look at a map of the area in which are 
specified all the farms. It is also shown if the farm's owner has performed 
a good job in that period.
At the end of each analysis the policy makers 
update the map (they are the only ones that are able to modify it).



\subsubsection{Interaction between farmers}
This functionality permits the farmers to comunicate with each other. 
The application has a specific web page were the farmers can send messages 
whenever they want.
If a farmer has an issue, before submitting a formal request of help to 
the policy makers 
by their home page, he can ask informally an advice by sending a message 
in the Forum.
It is not necessary to be a good farmer in order to answer someone elses 
message. All the messages are visible to everyone and 24h. 
Since it is an online application an internet connection is required to read or write 
on this page.



\subsection{Scenario}

\subsection{User Characteristics}
Dream has two different custumers that need to be 
distinguished in order to provide the various 
features specified in subsection ~\ref{subsection:2.2}. 
We give for granted that our users have interent access.
\subsubsection{Farmer}
\subsubsection{Policy Maker}


\subsection{Assumptions, Dependencies and Constraints}
\textbf{we focus on any limitation we should face but also the domain assumption 
(what we should assume in order for our system to offer the service as expected, 
what the external environment in order to fulfill our general goals)}

\section{Specific Requirements}

\section{Formal Analysis using Alloy}


\end{document}

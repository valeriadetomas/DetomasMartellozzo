\subsection{Purpose}
The purpose of this document is to thoroughly describe 
Data-dRiven PrEdictive FArMing in Telengana(DREAM).
It presents functional and non functional requirements of the system and its components.
Moreover it provides use cases and scenarios for the users involved.
\\This document is meant as a contractual basis for the customer and the developer.

\subsubsection{Goals}\label{section:1.1.1}
  \begin{table}[h]
        \rowcolors{1}{gray!25}{white}  
        \centering
  \begin{tabular}{|p{2cm}|p{9cm}|}
    \hline
    G1 & allow policy makers to retrieve information from farmers and to evaluate their performance\\
    \hline
    G2 & allow farmers to communicate with each other\\
    \hline
    G3 & allow farmers to insert data and advices on his production\\
    \hline
    G4 & allow farmers to send request of help to Policy Makers\\
    \hline
    G5 & allow farmers to retrieve information relevant for their activity\\
    \hline
    G6 & Policy Makers and farmers should be able to consult the map of the zone ( and the informations stored in it ) with different levels of visibility\\
    \hline
    G6A & Policy Makers and farmers should be able to see the position of the farms on the map\\
    \hline
    G6B & Policy Makers should be able to see both the information about the production and the evaluation of each farm\\
    \hline
    G6C & farmers should be able to see only the type of production of a farmer by the map\\
    \hline
  \end{tabular}
  \caption{Definition of goals}
\end{table}

\subsection{Scope}

The aim of the system is to acquire and combine data and information of farmers in Telengana. 
The system will provide support both to Telengana’s policy makers and to farmers 
thanks to new innovative technologies.\\
Through the system, policy makers are able 
to get a complete picture of the agriculture status in the whole state. In order 
to obtain this, Dream provides information that make policy makers able to give 
incentives to those farmers who are performing well. Moreover it allows them to 
keep track of those farmers who need help. \\
The farmers have access to a forum where they are able to communicate with other 
farmers. The aim of the forum is to share useful suggestions and to let farmers 
who struggle with something ask for help. \\
Therefore on the one side policy makers can have an entire perspective of the farms’ 
situation in the entire state and on the other side farmers can take advantage of the 
application and discuss with their colleagues.
\par 
Hence the application is used by the policy makers as a way to monitor farmers. 
Through the system they are able to search a farm and have access to its general information. 
Once a month they salso end each farmer an evaluation message where they specify if the farmer activity was
performed good or bad.\\
On the contrary farmers can ask for help or write advices. They are free to write messages and communicate with other farmers through a forum 
that is created especially for them.




\subsubsection{World Phenomena}
\begin{itemize}
    \item Farmers decide the type of production
    \item Weather conditions influence production
    \item Farmers will receive some type of advantage if their performance is evaluated godd enough
    \item 
    \item farmers can be identified as those who are performing well or not.
    \item 
\end{itemize}

\subsubsection{shared phenomena}

\begin{itemize}
    \item Humidity of soil is measured by sensors 
    \item The amount of water used by each farmer is retrieved by the water irrigation system
    \item Telengana's governments collects data concerning weather forecast
    \item Farmers insert data about their production in the system
    \item Farmers can insert problems they face into the system
    \item Farmers can answer to requests for help from other farmers
    \item Farmers can discuss with each other through the system
    \item The system identifies the farmers who are performing well
\end{itemize}

\subsection{Glossary}
\subsubsection{Definitions}
\subsubsection{Acronyms}
\subsubsection{Abbreviations}

\subsection{Document Structure}
\begin{enumerate}
    \item \textbf{Introduction}\\
            This section offers an introduction and a brief overview of the system that is presented in the document. 
            It highlights the purpose of the system and the goals that are meant to be achieved with it. 
            At the end there is also a glossary that contains a list of definitions, acronyms and abbreviations.
            
    \item \textbf{Overall Description}\\
            This section starts with a product perspective that contains a description of the system's domain through a class diagram. 
            It includes also state diagrams which are used to give more details about the behavior of some objects in the model.
            The section contains also a clear description of the features offered by the system, 
            it identifies the actors involved and it describes their characteristics.
            At the end there are domain assumptions and general constraints.
            
    
    \item \textbf{Specif Requirements}\\
            This section enters into the details on how the system interacts with the external world. It describes 
            the interfaces that are required and offered through several visual mockups. 
            Moreover the section provides functional and nonfunctional requirements. Functional
            requirements are additionally described by use cases, sequence diagrams and scenarios.
            At the end the section focuses on nonfunctional requirements and various limitations that the system might face.

    \item \textbf{Formal Analysis using Alloy}\\
            This section provides the model described through Alloy language.
            
    \item \textbf{Effor Spent}\\
            This section has a record of the hours spent to complete this document.

\end{enumerate}


